\chapter{\label{ch:ch01}ТЕОРИТИЧЕСКАЯ ЧАСТЬ}
В данной главе приведены используемые программные инструменты, а также программа, взятая за основу для написания собственной.

\section{\label{sec:ch01/sec01}Язык Python}
Python представляет собой популярный высокоуровневый язык программирования, который предназначен для создания приложений различных типов. Это и веб-приложения, и игры, и настольные программы, и работа с базами данных. Довольно большое распространение питон получил в области машинного обучения и исследований искусственного интеллекта.
Для изучения этого языка весьма полезной придётся книга "Python. Подробный справочник"~\cite{book-Python} и, так как темой проекта является игра, также используется книга "УЧИМ PYTHON, ДЕЛАЯ КРУТЫЕ ИГРЫ"~\cite{book-Python_2}.

%сайт pythonpip.ru про Python и tkinter ~\cite{Python-tkinter-2} и сайт-курс по tkinter ~\cite{Course-tkinter}.

\section{\label{sec:ch01/sec02}Библиотека tkinter}
Tkinter~\cite{Python-tkinter} - библиотека, позволяющая создавать приложения с графическим интерфейсом.

Преимуществами Tkinter считаются встроенность в стандартный пакет Python (не требует отдельной установки) и кроссплатформенность, позволяющая писать приложения для разных операционных системах. Визуальные элементы отображаются как элементы текущей системы, поэтому приложения выглядят так, будто принадлежат конкретной платформе. Хорошим источником для изучения библиотеки послужит курс по Tkinter~\cite{Course-tkinter}. А для понимания, как пишутся именно игры на Tkinter, была использована книга Г.Н.Гутмана "Python. Библиотека Tkinter"~\cite{book-tkinter-games}.

На протяжении всей работы мне придется неоднократно  обращаться к справочнику по tkinter на сайте Metanit.com ~\cite{tkinter-manual}. Здесь кратко описаны все доступные виджеты и их параметры, а также приведены примеры использования.

\section{\label{sec:ch01/sec02}Geany}
Geany~\cite{geany} - это мощный, стабильный и легкий текстовый редактор для программистов, который предоставляет массу полезных функций, не усложняя рабочий процесс. Он работает на Linux, Windows и macOS, переведен на более чем 40 языков и имеет встроенную поддержку более чем 50 языков программирования.

\section{\label{sec:ch01/sec02}Модуль random}
Модуль Python, который генерирует случайные числа. Они не являются по-настоящему случайными.

Этот модуль может использоваться для выполнения случайных действий, таких как генерация случайных чисел, случайный вывод значений для списка, строки или др. Является встроенной библиотекой Python.

\section{\label{sec:ch01/sec02}Модуль time}
Модуль времени Python - это мощный инструмент для работы с операциями, связанными с временем, в Python. Он предоставляет функции для измерения временных интервалов, форматирования и анализа строк времени и даты, а также обработки часовых поясов. С помощью модуля времени вы можете легко работать со значениями времени и даты и выполнять широкий спектр операций, связанных со временем, в своем коде на Python.

\section{\label{sec:ch01/sec02}Прототип, взятый за основу проекта.}
Одной из первых игр, представляющих собой узнаваемую игру "Танчики", была "Battle City"~\cite{battle-city}.

Battle City — компьютерная игра для игровых приставок Famicom и Game Boy. В России и странах СНГ выпускалась на пиратских картриджах как в оригинальном виде, так и в модификации Tank 1990, и известна под неофициальным названием «Та́нчики»[1]. Её предшественником была аркадная игра Tank Battalion, выпущенная фирмой Namco в 1980 году.

Полигон действий виден сверху. Игрок должен, управляя своим танком, уничтожить все вражеские танки на уровне, которые постепенно появляются вверху игрового поля. Враги пытаются уничтожить штаб игрока (внизу игрового поля в виде орла) и его танк. На каждом уровне нужно уничтожить двадцать единиц бронетехники противника разных видов. Если противник (или игрок) сможет разрушить штаб или лишит игрока всех жизней — игра окончена.