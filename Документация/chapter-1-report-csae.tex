\chapter{\label{ch:ch01}ГЛАВА 1: ТЕОРИТИЧЕСКАЯ ЧАСТЬ} % Нужно сделать главу в содержании заглавными буквами

\section{\label{sec:ch01/sec01}Раздел 1: Изучение языка Python и библиотеки tkinter}

Мне потребовалось освежить и немного расширить свои знания по языку Python. Для этого я воспользовался книгами по Питону "Python. Подробный справочник"~\cite{book_Python} и "УЧИМ PYTHON, ДЕЛАЯ КРУТЫЕ ИГРЫ"~\cite{book_Python_2}.

Затем я рассмотрел основы написания программ с использованием библиотеки tkinter ~\cite{Python-tkinter},
ещё мне помог сайт pythonpip.ru про Python и tkinter ~\cite{Python-tkinter_2} и сайт-курс по tkinter ~\cite{Course-tkinter}.

\section{\label{sec:ch01/sec02}Раздел 2: Изучение кода программ других разработчиков}

Чтобы разобраться, как пишутся именно игры на Python, я должен был рассмотреть конкретные примеры. Сайт github предоставил прекрасную возможность изучить простенькие игры.

%При помощи этой программы уже можно заметить основную структуру однофайловой программы, которую я позаимствую для написания своей программы.

\section{\label{sec:ch01/sec03}Раздел 3: Задумка и основные функции}

Моей программой должна стать игра "Танчики". Это должно быть, в первую очередь, игровое поле, сосотоящее из блоков (кирпичных и железных стен). В ней игроку предстоит перемещаться по игровом полю и побеждать вражеские танки, за уничтожение танка начисляются очки, которые формируют счет игрока. Игра продолжается до тех пор, пока танк игрока не будет уничтожен.

Само игровое поле будет иметь возможность редактировать размер, состав и расположение объектов на ней, при это будет менять и размер окна программы.

\section{\label{sec:ch01/sec04}Раздел 4: Структура программы}

В данном разделе будут описаны все составляющие игры.

\subsection{\label{sec:ch01/sec04/sub01}Игровое меню}

Игровое меню - это то, с чего начинается программа сразу после запуска. Это будет окно размером 800*600 пикселей, содержащий на себе название игры и список кнопок, по которым будет происходить переход.

\subsection{\label{sec:ch01/sec04/sub02}Настройки}

Один из переходов в программе, к которой можно перейти из главного меню. Он будет содержать такие опции, как возможность создать свое собственное поле с желаемыми размерами и разместить на них игровые объекты (кирпичные, железные блоки) либо выбрать поле по умолчанию.

\subsection{\label{sec:ch01/sec04/sub03}Окно "О разработчике"}

Это окно будет выводиться по нажатии кнопки "Разработчик" в главном меню. Оно будет содержать информацию Фамилии, Имени и группы студента, разрвботавшего данную программу.

\subsection{\label{sec:ch01/sec04/sub04}Игровое поле}

Это онсновной цикл в программе, в котором на холсте (окне игры) будут размещаться игровые объекты, происходить взаимодействия между объектами, игроку будет предоставлено управление с клавиатуры.

Размер окна в этом разделе будет варьироваться от количества блоков игрового поля в ширину и высоту (количество блоков * размер блока) (но не равны нулю, само собой).